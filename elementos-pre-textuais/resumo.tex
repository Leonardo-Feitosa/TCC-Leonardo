A tecnologia de acesso sem fio local tem sido uma das mais amplamente adotadas no mundo, tanto em termos de dispositivos compatíveis quanto de infraestrutura. Seu sucesso pode ser explicado por um aumento de performance que satisfaz os usuários que precisam de largura de banda, simplicidade para acessar a rede em todos os lugares, chegar a locais onde a rede cabeada não consegue e a possibilidade dos usuários serem móveis. Entre as tecnologias existentes, as WLANs padrão IEEE 802.11 destacam-se como as escolhas mais populares por empresas e usuários domésticos. Mas, para que estas redes operem adequadamente e coexistam em harmonia, é fundamental a adoção de procedimentos aplicados no planejamento ou avaliação das mesmas. O presente trabalho apresenta um estudo elaborado na rede Wi-Fi implantada no Bloco Didático do IFCE \textit{campus} Tauá abrangendo os seus ambientes fechados e abertos. Para isso, baseando-se na pesquisa bibliográfica realizada, optou-se pela escolha da ferramenta \textit{site survey} como metodologia para análise de desempenho de redes sem fio. Através dos mapas de calor gerados ao fim da execução do \textit{site survey}, característicos desta metodologia, foi possível identificar os locais onde o sinal de rádio apresentava maior e menor cobertura efetiva. A partir dos resultados obtidos foram propostas soluções para melhoria de qualidade do sinal Wi-Fi no Bloco Didático.

% Separe as palavras-chave por ponto
\palavraschave{Redes sem fio. Internet. 802.11. Wi-Fi. Site survey.}