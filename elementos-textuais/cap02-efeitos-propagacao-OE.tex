\chapter{Transmissão em redes sem fio}
\label{cap:transmissao-redes-sem-fio}

\section{O Espectro Eletromagnético}
\label{sec:espectro-eletromagnetico}

Quando se movem, os elétrons criam ondas eletromagnéticas que podem se propagar pelo espaço livre (até mesmo no vácuo), transportando energia durante o percurso. Essas ondas foram previstas pelo físico inglês James Clerk Maxwell em 1865 e foram observadas pela primeira vez pelo físico alemão Heinrich Hertz em 1887 \cite{tanenbaum2011}. O número de ciclos (oscilações) por segundo de uma onda eletromagnética é chamado frequência, e é medido em Hertz (Hz) -- em homenagem a Heinrich Hertz. A distância entre dois pontos máximos (ou mínimos) consecutivos é chamada comprimento de onda, designada universalmente pela letra grega $\lambda$ (lambda).

A velocidade de uma onda qualquer depende do meio em que ela se propaga. No vácuo, todas as ondas eletromagnéticas viajam na mesma velocidade, a da luz, que é aproximadamente igual a $3 \times 10^8$ m/s, independentemente de sua frequência \cite{tanenbaum2011}.

Essas três grandezas -- frequência ($f$), comprimento de onda ($\lambda$) e velocidade ($c$) -- se relacionam (no vácuo) através da seguinte expressão:

\begin{equation}
	\begin{aligned}
		c = \lambda f
	\end{aligned}
\end{equation}

\begin{figure}[H]
	\centering
	\Caption{\label{fig:onda}Propriedades de uma onda.}	
	\UECEfig{}{
		\includegraphics[scale=.4]{figuras/onda0.pdf}
	}{
		\Fonte{\citeonline[p. ~10]{flickenger2008}.}
	}	
\end{figure}

Uma vantagem da onda eletromagnética é o fato de que ela pode ser gerada ou captada por circuitos eletrônicos simples. Além disso, esse tipo de onda se propaga no vácuo, o que permite a comunicação entre antenas terrestres com satélites no espaço e vice-versa, entre os próprios satélites e entre dois pontos localizados em qualquer parte da superfície terrestre \cite{fluminense2010}.

Os dispositivos sem fio são restritos para operar em uma determinada banda (ou faixa) de frequência. Cada faixa tem uma largura de banda associada, que é simplesmente a quantidade de espaço de frequência na banda. Se a variação entre 2,40 GHz e 2,48 GHz é usada por um dispositivo, então a largura de banda será de 0,08 GHz ou 80 MHz.

A largura de banda adquiriu uma conotação de ser uma medida da capacidade de dados de um \textit{link}. Uma grande quantidade de matemática, teoria da informação e processamento de sinais pode ser usada para demonstrar que fatias mais largas de frequência podem ser usadas para transmitir mais informações \cite{gast2002}. Por exemplo, um canal de telefonia móvel analógico requer uma largura de banda de 20 kHz. Os sinais de televisão são muito mais complexos, já que, necessariamente, transmitem tráfego de áudio e vídeo, e por esse motivo possuem uma largura de banda consideravelmente maior, cerca de 6 MHz \cite{gast2002}.

O uso de uma faixa de frequência é rigorosamente controlado pelas autoridades reguladoras através de processos de licenciamento. No âmbito mundial, o processo de padronização de alocação de frequências para uso específico é realizado pela União Internacional de Telecomunicações (ITU, do inglês, \textit{International Telecommunication Union}). No Brasil, a ANATEL (Agência Nacional de Telecomunicações) representa a entidade responsável pela definição e fiscalização da utilização das faixas de frequência em território nacional. Essas determinações regulatórias visam coibir o uso das faixas de frequência sem permissão por infratores como estações de rádio e TVs piratas.

Entretanto, existem faixas de frequência que não estão sujeitas a autorização de uso pelos órgãos reguladores, ou seja, são bandas de frequência abertas para transmitir. Essas frequências não licenciadas são conhecidas como ISM (do inglês, \textit{Industrial, Scientific, Medical}). As bandas ISM foram padronizadas na maioria dos países em três faixas de frequência: 900 MHz, 2,4 GHz e 5 GHz \cite{moraes2010,tanenbaum2011}. A figura abaixo exibe a alocação das frequências ISM e também das bandas U-NII (do inglês, \textit{Unlicensed National Information Infrastructure}).

\begin{figure}[H]
	\centering
	\Caption{\label{fig:ism_unii}As bandas ISM e U-NII.}	
	\UECEfig{}{
		\includegraphics[scale=.6]{figuras/ism-unii.pdf}
	}{
		\Fonte{\citeonline[p. ~66]{tanenbaum2011}.}
	}	
\end{figure}

Cada banda de frequência utilizada nas telecomunicações estão contidas em um modelo de escala comum, onde é apresentado o intervalo completo de todas as possíveis frequências da radiação eletromagnética, denominado de espectro eletromagnético. A figura abaixo ilustra todas as variações de frequências contidas no espectro eletromagnético.

\begin{figure}[H]
	\centering
	\Caption{\label{fig:espectro}O espectro eletromagnético e a maneira como ele é usado na comunicação.}	
	\UECEfig{}{
		\includegraphics[scale=.72]{figuras/espectro_eletromagnetico.pdf}
	}{
		\Fonte{\citeonline[p. ~70]{tanenbaum2011}.}
	}	
\end{figure}

\begin{citacao}
	As faixas de rádio, microondas, infravermelho e luz visível do espectro podem ser usadas na transmissão de informações, por meio de modulação da amplitude, da frequência ou da fase das ondas. A luz ultravioleta, os raios X e os raios gama representariam opções ainda melhores, por terem frequências mais altas, mas são difíceis de produzir e modular, além de não se propagarem bem através dos prédios e de serem perigosos para os seres vivos \cite{tanenbaum2011}.
\end{citacao}

\section{Efeitos de Propagação em Ondas Eletromagnéticas}
\label{sec:efeitos-propagacao-OE}

Sistemas de comunicações sem fio utilizam-se de ondas eletromagnéticas para o envio de sinais através do ar. Na perspectiva de um usuário, conexões sem fio não são particularmente diferentes de qualquer outro tipo de conexão de rede: os serviços de transmissão de informações funcionarão de acordo com o esperado \cite{flickenger2008}. Mas ondas de rádio possuem algumas propriedades inesperadas se comparadas com o meio guiado.  Por exemplo: é muito fácil ver o caminho que o cabo Ethernet faz, só é preciso seguí-lo em sua extensão. Também pode-se ter a confiança de que ter vários cabos Ethernet lado a lado não causarão problemas, uma vez que os sinais trafegam no interior dos fios.

Diferentemente dos enlaces físicos, o caminho de uma onda de rádio entre transmissor e receptor pode variar de uma simples linha de visão completamente desobstruída até um cenário em que seja obstruído por prédios, terrenos elevados e áreas de vegetação densa \cite{rappaport2009}. Isso quer dizer que os sinais de rádio são aleatórios e de difícil análise. Até a velocidade do deslocamento dos terminais influencia na rapidez com que o sinal enfraquece \cite{rappaport2009}.

Portanto, o estudo da propagação dos sinais de radiofrequência é importante para a compreensão das comunicações sem fio porque fornece a modelagem física necessária, o que resulta em uma boa estimativa de potência requerida para o estabelecimento do enlace de comunicação para que haja comunicação confiável \cite{haykin2009}. Além disso, o estudo da propagação auxilia na compreensão das técnicas de recepção para compensação das perdas introduzidas pela transmissão sem fio \cite{haykin2009}.

Os efeitos sofridos pela onda eletromagnética ao se propagar são diversos, mas os principais e mais importantes são a reflexão, a difração, a refração, a absorção, o desvanecimento e a interferência \cite{flickenger2008,haykin2009,rappaport2009}.

\subsection{Absorção}
\label{sub:absorcao}

Quando ondas eletromagnéticas penetram algum objeto, elas geralmente atenuam ou dissipam-se totalmente. O quanto elas perdem de potência irá depender de sua frequência e, claro, do material em que penetram \cite{flickenger2008}. Em analogia, janelas de vidro são, obviamente, transparentes para a luz, enquanto o vidro usado em óculos de sol filtram uma boa quantidade da intensidade da luz e também da radiação ultravioleta. 

\begin{figure}[H]
	\centering
	\Caption{\label{fig:absorcao}Atenuação devido a absorção.}	
	\UECEfig{}{
		\includegraphics[scale=.78]{figuras/absorcao1.png}
	}{
		\Fonte{Adaptado de MegaSorber.}
	}
\end{figure}

Para ondas de rádio, os dois principais materiais absorventes são \cite{flickenger2008}:

\begin{alineas}
	\item Metal: elétrons podem mover-se livremente em metais, sendo prontamente capazes de oscilar e absorver a energia de uma onda que incida sobre eles;
	\item Água: ondas de rádio fazem com que as moléculas de água agitem-se, tomando parte da energia da onda.
\end{alineas}

Em termos práticos de redes sem fio, podemos considerar os metais e a água como excelentes absorventes: as ondas de rádio não serão capazes de atravessá-los com facilidade \cite{flickenger2008}.

\subsection{Reflexão}
\label{sub:reflexao}

Nas comunicações sem fio terrestres, normalmente não existe uma linha de visada desimpedida no percurso do sinal de rádio entre transmissor e receptor e as comunicações geralmente envolvem o fenômeno da reflexão (HAYKIN, 2009).

\begin{citacao}
	A reflexão ocorre quando uma onda eletromagnética em propagação colide com um objeto que possui dimensões muito grandes em comparação com o comprimento de onda da onda que se propaga  \cite{rappaport2009}.
\end{citacao}

\begin{figure}[H]
	\centering
	\Caption{\label{fig:reflexao}Reflexão.}	
	\UECEfig{}{
		\includegraphics[scale=.65]{figuras/reflexao.pdf}
	}{
		\Fonte{\citeonline[p. ~18]{flickenger2008}.}
	}
\end{figure}

\subsubsection{Multipercurso}
\label{subsub:multipercurso}

Ainda que as regras de reflexão sejam relativamente simples, a situação pode se complicar quando se imagina o interior de um escritório com várias divisórias e objetos metálicos das mais variadas formas e tamanhos. O mesmo se aplica aos ambientes urbanos, já que estes apresentam construções de concreto, árvores, veículos e um intenso fluxo de pessoas circulando pelas ruas.

Nessas áreas densamente ocupadas, a maior parte da comunicação acontece por espalhamento das ondas eletromagnéticas ao chocarem-se contra a superfície das construções e objetos ao redor \cite{haykin2009}.  Esses percursos de propagação múltiplos são conhecidos como multipercurso ou multicaminho. Até mesmo quando existe uma linha de visão direta, o multipercurso ainda ocorre devido às reflexões no solo e nas estruturas próximas a estação móvel \cite{rappaport2009}.

\begin{figure}[H]
	\centering
	\Caption{\label{fig:multipath}Sinais refletidos.}	
	\UECEfig{}{
		\includegraphics[scale=.8]{figuras/Multipath1.png}
	}{
		\Fonte{Adaptado de Yatebts.}
	}
\end{figure}

\subsection{Difração}
\label{sub:difracao}

\begin{citacao}
	A difração ocorre quando o caminho de rádio entre o transmissor e o receptor é obstruído por uma superfície que possui irregularidades afiadas (arestas). As ondas secundárias resultantes da superfície de obstrução estão presentes pelo espaço e até mesmo por trás do obstáculo, fazendo surgir uma curvatura de ondas em torno do obstáculo, até mesmo quando não existe um caminho de linha de visão entre transmissor e receptor \cite{rappaport2009}.
\end{citacao}

O fenômeno da difração, de maneira sucinta, está relacionado ao fato de as ondas eletromagnéticas contornarem objetos quando passam ao redor dos mesmos, tais como arestas de construções ou picos de montanhas ou quando atravessam barreiras contendo aberturas. Para altas frequências, a difração, assim como a reflexão, depende do formato do objeto, além da amplitude, fase e polarização da onda incidente sobre o ponto difrator \cite{rappaport2009}.

\begin{figure}[H]
	\centering
	\Caption{\label{fig:difracao}Difração sobre o topo de uma montanha.}	
	\UECEfig{}{
		\includegraphics[scale=.62]{figuras/difracao1.pdf}
	}{
		\Fonte{\citeonline[p. ~20]{flickenger2008}.}
	}
\end{figure}

Vale ressaltar que a difração ocorre ao custo de perda de potência, isto é, a energia da onda difratada é significativamente menor que a da onda original. Mas ainda sim existe, nas ondas secundárias, força suficiente para produzir um sinal útil, além da vantagem da difração para contornar obstáculos \cite{flickenger2008,rappaport2009}.

\subsection{Refração}
\label{sub:refracao}

A refração ocorre quando as ondas eletromagnéticas mudam a trajetória de propagação quando passam de um meio para outro. Na transição, o nível de energia da onda é reduzido, pois uma fração da onda é refletida \cite{flickenger2008}. Um exemplo de refração ocorre quando a luz, propagando-se no ar, encontra uma interface com a água. A utilização da refração nas comunicações sem fio fica limitada a circunstâncias especiais, tais como comunicações via satélite, pois a transmissão necessita penetrar através das camadas da atmosfera, cada uma com densidade distinta da outra \cite{rappaport2009}.

\begin{figure}[H]
	\centering
	\Caption{\label{fig:refracao}Refração.}	
	\UECEfig{}{
		\includegraphics[scale=.65]{figuras/refracao.pdf}
	}{
		\Fonte{Adaptado de AlunosOnline.}
	}
\end{figure}

\subsection{Interferência}
\label{sub:interferencia}

Interferência é o fenômeno que corrompe um sinal enquanto ele viaja em um enlace da origem para o destino. A perturbação pode interromper, obstruir, degradar ou limitar a recepção efetiva de sinais. Esses efeitos podem variar de uma simples degradação de dados a uma perda total de dados. O termo é frequentemente usado para se referir à adição de sinais indesejados a um sinal útil \cite{flickenger2008}.

Em comunicações sem fio, a interferência é causada principalmente por fontes de radiofrequência que operam na mesma faixa de frequência, como por exemplo, aparelho de microondas e redes Wi-Fi, ambas operando na banda de 2,4 GHz \cite{moraes2010}.

\begin{figure}[H]
	\centering
	\Caption{\label{fig:interferencia}Interferência de radiofrequência.}	
	\UECEfig{}{
		\includegraphics[scale=.88]{figuras/Interference1.png}
	}{
		\Fonte{Adaptado de Yatebts.}
	}
\end{figure}

\subsection{Desvanecimento (\textit{fading})}
\label{sub:desvanecimento}

O desvanecimento é um fenômeno causado pela variabilidade da intensidade do sinal no tempo associado à mobilidade da estação móvel \cite{haykin2009,rappaport2009}. Frequentemente, o sinal recebido é uma combinação de vários modos de propagação, resultantes da reflexão e da difração (detalhados anteriormente). Assim, a maior parte da comunicação acontece por espalhamento das ondas eletromagnéticas, que chegam de diferentes direções com diferentes atrasos de propagação \cite{haykin2009}. No receptor, o sinal final é a soma vetorial dessas ondas de caminhos múltiplos, podendo interagir umas com as outras construtiva ou destrutivamente, dependendo da amplitude e da fase de cada componente espectral \cite{haykin2009}.

\begin{figure}[H]
	\centering
	\Caption{\label{fig:fading}Interferência construtiva e destrutiva.}	
	\UECEfig{}{
		\includegraphics[scale=.4]{figuras/inter.png}
	}{
		\Fonte{\citeonline[p. ~57]{haykin2009}}
	}
\end{figure}

A rapidez com que as flutuações na magnitude do sinal ocorre pode ser classificada como desvanecimento lento ou desvanecimento rápido \cite{haykin2009}:

\begin{citacao}
	Desvanecimento lento surge do fato que a maioria dos grandes refletores e objetos difratores ao longo do percurso de transmissão estão distante do terminal. O movimento do terminal relativamente a esses objetos distantes é pequeno; consequentemente, as mudanças correspondentes na propagação são sentidas muito lentamente, Esses fatores contribuem para as perdas do percurso médio entre um transmissor fixo e um receptor fixo \cite{haykin2009}.
	
	Desvanecimento rápida surge da variação rápida dos níveis de sinal quando o terminal usuário move-se em distâncias curtas. O desvanecimento rápido é causado por reflexões de objetos e pelo movimento do terminal relativamente a esses objetos \cite{haykin2009}.
\end{citacao}

\section{Modelos de Propagação}
\label{sec:modelos-propagacao}

Para que se possa realizar um projeto de enlace de rádio confiável e com boa eficiência, são utilizados os chamados modelos de propagação. Os mesmos são desenvolvidos com base em medições empíricas que buscam alimentar com dados todo um processo matemático complexo capaz de representar a proliferação das ondas de rádio, predizer a perda de caminho ou a cobertura efetiva de um transmissor \cite{akpaida2018,najnudel2004}. Assim, é fácil concluir que, quanto mais informações for possível representar nestas equações, mais precisa será a caracterização do meio e seus efeitos \cite{akpaida2018}.

Modelos de propagação de rádio são empíricos por natureza. E como todos os modelos empíricos, os modelos de propagação de rádio não chamam a atenção para a conduta exata de uma conexão, mas sim para prever o comportamento que o \textit{link} pode mostrar sob as condições especificadas \cite{akpaida2018}.

Existem diversos modelos de propagação formulados por estudiosos e organizações voltadas para o ramo das comunicações sem fio. Cada modelo aplica-se a uma determinada situação específica, dependendo da característica física do local (região urbana ou região rural, por exemplo); frequência de operação do sistema de comunicação e o tipo de material da construção, fator esse, crítico para as redes Wi-Fi.

\subsection{Modelo de propagação no espaço livre}
\label{sub:espaco-livre}

O modelo de propagação no espaço livre é usado para prever a intensidade do sinal recebido quando o transmissor e o receptor possuem um linha de visão sem a presença de obstáculos, ou seja, quando há caminho livre entre eles \cite{rappaport2009}. Os sistemas de comunicação via satélite normalmente experimentam uma transmissão com caminho desobstruído.

O modelo de espaço livre, assim como a maioria dos modelos de propagação de radiofrequência, prevê que a potência recebida diminui em função da distância de separação transmissor-receptor (T--R) \cite{rappaport2009}. A potência no espaço livre recebida por uma antena receptora que está separada de uma antena transmissora, irradiando, por uma distância T--R é dada pela equação do espaço livre:

\begin{equation}
	\begin{aligned}
	\label{eq:friis}
		P_R = \dfrac{P_TG_TG_R}{L_P}
	\end{aligned}
\end{equation}

\noindent onde $P_T$ é a potência transmitida, $P_R$ E a potência recebida, $G_T$ é o ganho da antena transmissora, $G_R$ é o ganho da antena receptora e $L_P$ é a perda do percurso entre T--R. Tanto a dedução matemática do ganho ($G$) da antena quanto a da \emph{perda do percurso} ($L_P$) pode ser encontrada com detalhes em \citeonline{haykin2009}.

A equação \ref{eq:friis} é conhecida como \emph{equação de Friis}. É possível simplificá-la em função do ganho em decibel (dB) \cite{haykin2009}:

\begin{equation}
	\begin{aligned}
	\label{eq:friis-decibel}
		P_R(dB) = P_T(dB) + G_T(dB) + G_R(dB) - L_P(dB)
	\end{aligned}
\end{equation}

\noindent onde $X(dB) = 10\log_{10} (X)$. A \emph{equação de Friis} é a equação fundamental para o planejamento do enlace de rádio, pois relaciona as potências transmitida e recebida, considerando as condições da transmissão \cite{haykin2009}. Ela fornece os requisitos essenciais para que o nível de potência requerida pelo receptor seja suficiente para ele detectar as informações transmitidas com confiabilidade \cite{haykin2009}.

\subsection{Modelo de dois raios}
\label{modelo-2-raios}

A equação do espaço livre (Equação \ref{eq:friis}) considera que sempre existe uma linha de visada direta entre transmissor e receptor, não considera, entretanto, o efeito da superfície terrestre na comunicação. Esse fato raramente acontece com transmissões em solo, o que torna, o modelo de espaço livre impreciso na grande maioria dos casos \cite{rappaport2009}.

O modelo de reflexão no solo (Figura \ref{fig:2raios}), baseado na ótica geométrica, considera o caminho direto e um caminho de propagação refletido no solo entre o transmissor e o receptor \cite{rappaport2009}.

\begin{figure}[H]
	\centering
	\Caption{\label{fig:2raios}Modelo de reflexão no solo com dois raios.}	
	\UECEfig{}{
		\includegraphics[scale=.8]{figuras/modelo2raios.pdf}
	}{
		\Fonte{Adaptado de \citeonline[p. ~80]{rappaport2009}}
	}
\end{figure}

A potência recebida a uma distância $d$ do transmissor para o modelo de dois raios pode ser expressa como \cite{rappaport2009}:

\begin{equation}
	\begin{aligned}
	\label{eq:2-raios}
		P_R = P_TG_TG_R\dfrac{h^2_Th^2_R}{d^4}
	\end{aligned}
\end{equation}

\noindent onde $P_R$ é a potência recebida, $P_T$ é a potência transmitida, $G_T$ é o ganho da antena transmissora, $G_R$ é o ganho da antena receptora, $h_T$ é a altura do transmissor e $h_R$ é a altura do receptor. Todo o desenvolvimento algébrico até chegar a Equação \ref{eq:2-raios} pode ser encontrada em \citeonline{rappaport2009} ou \citeonline{haykin2009}.