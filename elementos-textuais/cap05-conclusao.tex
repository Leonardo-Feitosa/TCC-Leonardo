\chapter{Considerações Finais e Trabalhos Futuros}
\label{cap:conclusoes-e-trabalhos-futuros}

A popularidade massiva das redes sem fio tem feito com que o custo dos equipamentos caia contínua e rapidamente, enquanto a tecnologia empregada nos mesmos aumenta de forma acelerada. A propriedade do usuário ser móvel impulsionou a disseminação dos dispositivos móveis, possibilitando-os de se conectarem à Internet em qualquer local com cobertura disponível. Assim, permitir às pessoas o acesso fácil e barato à informação através da Internet é, acima de tudo, beneficiá-las diretamente com a democratização do conhecimento.

O desenvolvimento do presente estudo possibilitou a aplicação da metodologia \textit{site survey}, a qual pode contribuir para o aumento de desempenho da rede Wi-Fi do Bloco Didático do IFCE \textit{campus} Tauá. Além disso, também permitiu uma pesquisa de campo para obter dados mais consistentes sobre como ocorre a propagação de radiofrequência por todo o local de interesse.

A partir da pesquisa feita, a escolha do \textit{site survey} para a proposta do trabalho se deu graças ao seu alto grau de flexibilidade de execução tanto em redes cabeadas como em redes \textit{wireless}, voltada para a identificação de problemas. Observou-se que esta técnica pode ser empregada na avaliação de redes já em funcionamento e também no planejamento de um novo sistema, seja para pequenos projetos ou mesmo para projetos de maiores proporções.

O principal objetivo do trabalho foi alcançado, que foi utilizar as técnicas de \textit{site survey} na rede sem fio existente no Bloco Didático, pelo qual após a conclusão das etapas na rede \textit{wireless} IFCE\_Teste foi possível explorar as informações coletadas sobre o desempenho da rede Wi-Fi por toda a área. Com as informações processadas, gerou-se os mapas de calor que possibilitaram com que se identificasse as regiões com maior e menor qualidade de sinal recebido.

De acordo com os mapas de calor obtidos, constatou-se que, dos dois andares mapeados, o primeiro apresenta uma propagação de sinal pobre em relação ao segundo, onde a intensidade de potência é mais satisfatória, principalmente na área central e nas salas imediatamente próximas a coordenação.
Em relação a ferramenta de \textit{software} utilizada, o Ekahau HeatMapper teve um funcionamento livre de falhas graves, mesmo quando deixou de mostrar alguns pontos no mapa de calor do térreo do Bloco Didático. Mesmo sendo um programa gratuito, atendeu aos requisitos exigidos para a execução das atividades do \textit{site survey}.

Todos os resultados apresentados e discutidos neste trabalho, adquiridas por meio de um \textit{site survey}, demonstram que, devido a natureza complexa e aleatória da propagação de ondas de rádio, é importante definir métodos para avaliação/planejamento de redes sem fio que proporcionem resultados confiáveis com os quais possam ser utilizados para que se tenha uma melhor performance em redes do tipo Wi-Fi.

\section{Trabalhos Futuros}
\label{sec:trabalhos-futuros}

Como proposta de trabalhos posteriores, a metodologia \textit{site survey} pode ser implementada buscando novas linhas de pesquisa e estudo em projetos que envolvam redes sem fio, tais como:

\begin{compactitem}
	\item Realizar um novo \textit{site survey} no Bloco Didático para que mais parâmetros possam ser analisados, como tráfego de dados, interferências e a opinião dos usuários a respeito de aspectos da rede.
	
	%\item \textcolor{blue}{Aplicar o \textit{site survey} para a avaliação da nova rede sem fio implantada no Bloco Didático e comparar os resultados com os obtidos por este trabalho.}
	
	\item Expandir a aplicação do \textit{site survey} para todo o IFCE \textit{campus} Tauá, o que inclui o estacionamento (ambiente \textit{outdoor}) e o bloco principal (ambiente \textit{indoor}) para uma investigação mais completa.
\end{compactitem}


