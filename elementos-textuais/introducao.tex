\chapter{Introdução}
\label{cap:introducao}

Em 1971, o sistema Aloha\footnote{Originalmente, o sistema Aloha, desenvolvido por Norman Abramson em 1970, na Universidade do Havaí, funcionava como um protocolo para sistemas de comunicação via radiofrequência para o acesso remoto entre computadores enviando pacotes num sistema de radiocomunicação \cite{abramson1970acm, haykin2009}.} estabeleceu e operou uma rede de dados terrestres (AlohaNet) no estado americano do Havaí. Em 1973, ao sofrer uma adaptação, o sistema Aloha usou um transponder acoplada em um satélite experimental da Agência Espacial Americana (NASA, do inglês, \textit{National Aeronautics and Space Administration}) (ATS-1) para demonstrar uma rede internacional de dados via satélite conectando a NASA, na Califórnia, e cinco universidades nos Estados Unidos, Japão e Austrália, fornecendo a primeira demonstração pública de uma rede de dados por pacotes sem fio \cite{abramson1970acm, SchwartzAbramson2009ieee}. Também em 1973, o rede AlohaNet foi vinculada à ARPANet\footnote{A primeira rede de computadores por comutação de pacotes e uma ancestral direta da Internet pública de hoje \cite{kurose2013}.} (do inglês, \textit{Advanced Research Projects Agency Network}) \cite{abramson1970acm, SchwartzAbramson2009ieee}.

Mesmo com limitações como largura de banda e tecnologia de transmissão, foi graças ao pioneirismo do sistema Aloha, que as redes de comutação sem fio, conhecidas como \textit{wireless} ou ainda WLANs (do inglês, \textit{Wireless Local Area Network}) surgiram como redes complementares às redes cabeadas tradicionais. Essa tecnologia de comunicação se desenvolveu devido a necessidade de implementação de um método alternativo de conexão que não privasse a movimentação do usuário a uma posição fixa durante uma transmissão/recepção de dados na rede. O tempo passou e a tecnologia evoluiu, deixou de ser restrito ao meio acadêmico e militar e se tornou acessível a empresas e ao usuário doméstico. Nos dias de hoje podemos pensar em redes \textit{wireless} como uma alternativa bastante interessante em relação às redes cabeadas. Suas aplicações são muitas e variadas e o fato de ter a mobilidade como principal característica, tem facilitado a sua aceitação, principalmente nas empresas (FARIAS, 2005).

Com o advento de \textit{notebooks}, \textit{tablets}, \textit{smartphones} e a promessa de acesso desimpedido à Internet global de qualquer lugar por meio das comunicações sem fio, a qualquer hora, assistimos a uma explosão semelhante da utilização de dispositivos sem fio para acesso à Internet em comparação ao dos telefones móveis, não só através da rede de telefonia móvel, agora plenamente possível, mas também por meio das redes Wi-Fi. O uso deste tipo de rede está se tornando cada vez mais comum, não só nos ambientes domésticos e corporativos, mas também em locais públicos (bares, lanchonetes, shoppings, aeroportos, etc) e em instituições acadêmicas. Independentemente do crescimento futuro de equipamentos sem fio para Internet, já ficou claro que as redes sem fio e os serviços móveis relacionados que elas possibilitam vieram para ficar (KUROSE; ROSS, 2013; GAST, 2002).

No período em que as redes locais cabeadas (LAN -- \textit{Local Area Network}, do inglês)  dominavam a infraestrutura de rede de computadores, somente era possível conectar computadores à Internet e entre si por meio de cabos padrão Ethernet. Este tipo de conexão é bastante popular, mas conta com algumas limitações, por exemplo: só se pode movimentar o computador até o limite de alcance do cabo; ambientes com um grande número computadores podem exigir adaptações na estrutura do prédio para a passagem dos fios; em uma residência, pode ser necessário realizar perfurações na parede para que os cabos alcancem outros cômodos; a manipulação constante ou incorreta pode fazer com que o conector do cabo de rede se danifique. Com exceção da mobilidade, o restante das alegações implica em investimento financeiro direto por parte das empresas e também para o usuário doméstico. Felizmente, as redes sem fio Wi-Fi surgiram para eliminar estas limitações. Como consequência, alcançaram rapidamente uma disseminação generalizada tanto no espaço comercial quanto no residencial (ALECRIM, 2008).

%\lipsum[5]
%\lipsum[6]
%\lipsum[7]

\newpage
\section{Motivação}
\label{sec:motivacao}

%\lipsum[3]
%\lipsum[4]

\section{Objetivos}
\label{sec:objetivos}

Interdum et malesuada fames ac ante ipsum primis in faucibus. Lorem ipsum dolor sit amet, consectetur adipiscing elit. Ut ex tellus, sodales in euismod at, ultricies quis leo \cite{alecrim2008site}.

\subsection{Objetivo Geral}
\label{sec:objetivo-geral}

Integer imperdiet ac magna eu pulvinar. Aliquam erat volutpat. Etiam molestie, nulla a egestas aliquet, velit augue congue metus, et ultricies metus massa vel nibh. Vivamus viverra commodo finibus. Nam elementum convallis accumsan. Quisque tincidunt purus nisl, tincidunt ultricies odio ultrices eu.

\subsection{Objetivos Específicos}
\label{sec:objetivos-especificos}

Lorem ipsum dolor sit amet, consectetur adipiscing elit. Duis scelerisque, velit at facilisis hendrerit, dui eros lacinia metus, non maximus mi tortor ut lectus. Donec hendrerit leo ut consectetur tincidunt. 

	\begin{alineas}
		\item Lorem ipsum dolor sit amet, consectetur adipiscing elit. Nunc dictum sed tortor nec viverra.
		\item Praesent vitae nulla varius, pulvinar quam at, dapibus nisi. Aenean in commodo tellus. Mauris molestie est sed justo malesuada, quis feugiat tellus venenatis.
		\item Praesent quis erat eleifend, lacinia turpis in, tristique tellus. Nunc dictum sed tortor nec viverra.
		\item Mauris facilisis odio eu ornare tempor. Nunc dictum sed tortor nec viverra.
		\item Curabitur convallis odio at eros consequat pretium.
	\end{alineas}