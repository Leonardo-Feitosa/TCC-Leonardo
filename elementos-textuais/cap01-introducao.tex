\chapter{Introdução}
\label{cap:introducao}

\section{Contextualização}
\label{sec:contextualizacao}

Na década de 1970, o sistema Aloha\footnote{Originalmente, o sistema Aloha, desenvolvido por Norman Abramson em 1970, na Universidade do Havaí, funcionava como um protocolo para sistemas de comunicação via radiofrequência para o acesso remoto entre computadores enviando pacotes num sistema de radiocomunicação \cite{abramson1970acm,haykin2008}.} estabeleceu e operou uma rede de dados terrestres (AlohaNet) no estado americano do Havaí. O sistema usou um transponder acoplado em um satélite experimental da Agência Espacial Americana (NASA, do inglês, \textit{National Aeronautics and Space Administration}) (ATS-1) para demonstrar uma rede internacional de dados via satélite conectando a NASA, na Califórnia, e cinco universidades nos Estados Unidos, Japão e Austrália, fornecendo a primeira demonstração pública de uma rede de dados por pacotes sem fio \cite{abramson1970acm,SchwartzAbramson2009ieee}.

Mesmo com limitações como largura de banda e tecnologia de transmissão, foi graças ao pioneirismo do sistema Aloha, que as redes de comutação sem fio locais, conhecidas como LANs (do inglês, \textit{Local Area Network}) sem fio ou ainda WLANs (do inglês, \textit{Wireless Local Area Network}) surgiram como redes complementares às redes cabeadas tradicionais. Essa tecnologia de comunicação se desenvolveu devido a necessidade de implementação de um método alternativo de conexão que não privasse a movimentação do usuário durante a conexão com a rede. O tempo passou e a tecnologia evoluiu, deixou de ser restrita ao meio acadêmico e militar e se tornou acessível a empresas e ao usuário doméstico. Nos dias de hoje se pode pensar em redes \textit{wireless} (``sem fio'', em português) como uma alternativa bastante interessante em relação às redes cabeadas. Suas aplicações são muitas e variadas e o fato de ter a mobilidade como principal característica, tem facilitado a sua aceitação, principalmente nas empresas \cite{farias2005}.

Com o advento de \textit{notebooks}, \textit{tablets}, \textit{smartphones} e a promessa de acesso desimpedido à Internet global de qualquer hora e lugar por meio das comunicações sem fio, a demanda de acesso de dispositivos sem fio à Internet aumentou consideravelmente não só através da rede de telefonia móvel, agora plenamente possível, mas também por meio das redes Wi-Fi (do inglês, \textit{Wireless Fidelity}). O uso deste tipo de rede está se tornando cada vez mais comum, não só nos ambientes domésticos e corporativos, mas também em locais públicos (bares, lanchonetes, \textit{shoppings}, aeroportos, etc) e em instituições acadêmicas. Independentemente do crescimento futuro de equipamentos sem fio para Internet, já ficou claro que as redes sem fio e os serviços móveis relacionados que elas possibilitam se popularizaram e fazem parte do cotidiano das pessoas \cite{gast2002,kurose2013}.

Esse crescimento se deu, principalmente, com os avanços tecnológicos feitos na área, tornando a conexão mais rápida e confiável na troca de dados entre os nós da rede, fazendo com que os dispositivos móveis agregassem praticamente todas as funções que teria um computador de mesa convencional. Como consequência, estima-se que a força de trabalho móvel, em 2020, corresponderá a aproximadamente 1,75 bilhão de trabalhadores, segundo a Organização Internacional do Trabalho \cite{wba2017}. Para estes bilhões de usuários móveis, trabalhando dentro e fora do escritório tradicional, a mobilidade se tornou sinônimo de produtividade.

\textcolor{blue}{Para \cite{wba2017}, quando se trata de conexão sem fio com a Internet, os trabalhadores móveis preferem redes Wi-Fi ao invés de redes celulares, que são mais caras e com consumo de dados limitados. Dessa forma, o Wi-Fi transporta mais da metade de todos os dados móveis de acordo com o Índice de Rede Visual da Cisco (VNI, do inglês, \textit{Visual Networking Index}) \cite{wba2017}.}

\section{Justificativa}
\label{sec:justificativa}

\textcolor{blue}{Os sistemas de telecomunicação utilizam cada vez as tecnologias sem fio.} Como resultado, as formas tradicionais de trabalho em rede no mundo mostraram-se inadequadas para enfrentar os desafios apresentados por nosso novo estilo de vida coletivo. Se os usuários precisarem estar conectados a uma rede por cabos físicos, seu movimento será drasticamente reduzido. A conectividade sem fio, no entanto, não apresenta esta restrição e permite muito mais movimento livre por parte do usuário da rede. Como consequência, as tecnologias sem fio estão invadindo o domínio tradicional das redes ``fixas'' ou ``com fio'' \cite{gast2002}.

As redes sem fio compartilham várias vantagens importantes. A vantagem mais evidente da rede sem fio é a mobilidade. Usuários de redes sem fio podem se conectar às redes existentes e, em seguida, podem circular livremente. Por exemplo, um usuário de telefone celular pode percorrer quilômetros no decorrer de uma única conversa desde que ele esteja coberto por alguma torre celular \cite{gast2002}.

Da mesma forma, as redes Wi-Fi liberam os desenvolvedores de \textit{software} das amarras de um cabo Ethernet em uma mesa. Os desenvolvedores podem trabalhar na biblioteca, em uma sala de conferências, no estacionamento ou até mesmo na lanchonete do outro lado da rua. Enquanto os usuários sem fio permanecerem dentro do alcance do equipamento transmissor, eles poderão tirar proveito da rede \cite{gast2002}.

Em escolas e universidades, os laboratórios e bibliotecas estão quase sempre repletos de alunos que encontram na rede (Internet) uma boa fonte de consultas e pesquisas que complementam o conteúdo abordado em sala de aula, além de facilitar a comunicação e a troca de informações entre os estudantes. Graças à tecnologia de comunicação sem fio Wi-Fi, estudantes e professores não dependem exclusivamente dos laboratórios de informática para se conectarem à Internet.

O acesso sem fio pode ser disponibilizado por todo o ambiente educacional, bastando apenas o usuário ligar seu \textit{notebook} ou dispositivo móvel de qualquer local com cobertura e estabelecer a conexão com a rede.

%Outra propriedade benéfica das redes sem fio é que estas geralmente têm muita flexibilidade, o que pode se traduzir em implantação rápida. Redes sem fio usam um número de estações base para conectar usuários a uma rede existente. O lado da infraestrutura de uma rede sem fio, no entanto, é qualitativamente o mesmo, esteja conectando um usuário ou um milhão de usuários. Para oferecer serviço em uma determinada área, é necessário a presença de estações base e antenas no lugar.

%Uma vez que essa infraestrutura é construída, adicionar um usuário a uma rede sem fio é principalmente uma questão de autorização. Com a infraestrutura disponível, ela deve ser configurada para reconhecer e oferecer serviços aos novos usuários, mas a autorização não exige mais infraestrutura. Adicionar um usuário a uma rede sem fio é uma questão de configurar a infraestrutura, mas isso não envolve a passagem de cabos, o fechamento de terminais e a aplicação de correções em um conector \cite{gast2002}.

%Este exemplo simples ignora os desafios de escalabilidade. Naturalmente, se os novos usuários sobrecarregarem a infraestrutura existente, a própria infraestrutura precisará ser reforçada para que mais usuários possam ser conectados. A expansão da infraestrutura pode ser cara e demorada, especialmente se envolver aprovação legal e regulatória de faixas de frequência. No entanto, o ponto básico é válido: adicionar um usuário a uma rede sem fio pode ser reduzido a uma questão de configuração, enquanto adicionar um usuário a uma rede fixa requer conexões físicas \cite{gast2002}.

%Embora seja possível atender a um grupo fluido de usuários com conectores padrão Ethernet, o fornecimento de acesso através de uma rede com fio é problemático por vários motivos. A passagem de cabos é demorada, cara e também pode exigir adaptações no espaço físico; determinar corretamente o número de enlaces de cabos é difícil devido ao intenso fluxo de pessoas. Com uma rede sem fio, porém, não há necessidade de eventuais alterações na construção ou fazer suposições estatísticas complexas sobre a demanda. Uma infraestrutura com fio simples se conecta à Internet e, em seguida, a rede sem fio pode acomodar quantos usuários forem necessários \cite{gast2002}.

%Apesar das várias vantagens oferecidas pelas redes sem fio, as mesmas não substituem as LANs cabeadas. Servidores e outros equipamentos de \textit{datacenter} devem acessar dados, mas a localização geográfica do servidor é irrelevante. A velocidade das redes sem fio é limitada pela largura de banda disponível. A teoria da informação pode ser usada para deduzir o limite da velocidade de uma rede. A menos que as autoridades reguladoras estejam dispostas a aumentar as bandas de espectro não licenciadas, há um limite superior na velocidade das redes sem fio. O \textit{hardware} de rede sem fio tende a ser mais lento que o \textit{hardware} com fio. Diferentemente do padrão Ethernet, os padrões de rede sem fio devem avaliar cuidadosamente os quadros de entrada para proteger contra perda devido à falta de confiabilidade do meio \textit{wireless} \cite{gast2002}.

Para quantificar a relevância das redes \textit{wireless}, em 2022, 51\% do tráfego de dados móveis global será transportado pela redes Wi-Fi segundo a Cisco VNI no ano de 2018. No Brasil, 48\% do tráfego de dados total da Internet será movido através das redes Wi-Fi de acordo com as estimativas disponibilizadas também pela Cisco VNI em 2018.

Para que haja transferência de dados, as redes sem fio usam ondas de rádio como meio de transmissão, entretanto essa abordagem apresenta vários desafios. As ondas de rádio podem sofrer vários problemas de propagação que podem interromper o \textit{link} de rádio, como interferência, perdas de multipercurso e áreas de sombra (locais sem cobertura de sinal), que podem atenuar a potência do sinal Wi-Fi, reduzindo, por consequência, a taxa de transferência e a qualidade da conexão, assim como a viabilidade de alguns serviços que exigem um limite mínimo de destes recursos para funcionar com eficiência
 \cite{gast2002}. Isso significa que o canal de rádio impõe limitações fundamentais para o desempenho dos sistemas de comunicação sem fio.

% Proposta do trabalho
Para que uma WLAN tenha um desempenho satisfatório aos seus usuários, vários fatores devem ser levados em consideração para que o processo de instalação corresponda o mais fielmente possível ao planejado. Um método amplamente utilizado por projetista de redes, seja cabeada ou sem fio, é o \textit{site survey}. O \textit{site survey} é um conjunto de métodos de análise detalhada da estrutura do local onde será implantada a nova rede ou aplicada a uma infraestrutura já existente com o objetivo de identificar e solucionar problemas no sistema \cite{pinheiro2004site}.

Portanto, a proposta deste trabalho consiste em analisar a infraestrutura geral da rede sem fio do Bloco Didático do IFCE \textit{campus} Tauá a fim de avaliar se a configuração atual da rede sem fio provê um serviço satisfatório baseado na análise dos mapas de calor dos sinais de rádio da área.

Mediante o exposto, nesta proposta de trabalho será empregado a técnica de projeto chamada \textit{site survey} para levantar dados do Bloco Didático, especialmente sobre como ocorre a propagação dos sinais e medir o nível de sinal da rede sem fio, a partir dos pontos de acesso já presentes no local, pois estes são aspectos de grande impacto no desempenho das redes sem fio em geral.  Este método tem como intuito detectar possíveis problemas que incapacite a rede de prover a cobertura necessária para que serviços básicos (pesquisas, estudos, experimentos, etc.) dos estudantes, docentes e servidores sejam atendidas com o mínimo de satisfação.

É extremamente recomendado aliar ao \textit{site survey} ferramentas complementares para auxiliar o processo de inspeção do local, como, por exemplo, um \textit{software} adequado instalado em um \textit{notebook} que possa realizar a medição de potência do sinal e também identificar o canal de frequência em que opera as redes vizinhas, com objetivo de identificar possíveis interferências destrutivas no sinal.

Depois de fazer uso da ferramenta apropriada para a análise, é importante identificar o ambiente de trabalho para determinar a localização dos equipamentos e analisar o desempenho da rede Wi-Fi, para garantir uma melhor cobertura e prestar um melhor serviço.

Com os resultados se buscará simular o grau de eficiência da infraestrutura de rede sem fio ofertada pelo IFCE Tauá ao Bloco Didático, propor uma solução para conseguir um melhor desempenho através de recomendações para otimizar o serviço de rede \textit{wireless}.

\section{Objetivos}
\label{sec:objetivos}

Este trabalho apresenta como objetivo geral mostrar a aplicação da metodologia \textit{site survey} para análise de cobertura e recepção do sinal Wi-Fi no Bloco Didático do Instituto Federal do Ceará \textit{campus} Tauá.
Já com relação aos objetivos específicos, este trabalho visa:
\begin{compactitem}
	\item Analisar a infraestrutura atual da rede sem fio do Bloco Didático através de sua planta estrutural;
	\item Identificar e localizar os locais de concentração dos pontos de acesso para a elaboração de plantas, desenhos ou esquemáticos;
	\item Verificar a presença de possíveis obstáculos, fontes de interferência e áreas de sombra que possam limitar a propagação do sinal da rede sem fio;
	\item Coletar dados em campo do Bloco Didático a respeito da propagação do sinal da rede Wi-Fi;
	\item Sugerir uma melhoria para a rede com base nos resultados obtidos.
\end{compactitem}

\section{Organização do Trabalho}
\label{sec:organnizacao-do-trabalho}

Este trabalho está dividido da seguinte forma:

\begin{compactitem}
	\item Capítulo 2: aborda os principais conceitos por trás do processo de transmissão em redes sem fio, essenciais para a compreensão do funcionamento geral das comunicações sem fio, incluindo os mecanismos básicos da propagação de ondas eletromagnéticas e alguns modelos de propagação utilizados na predição de enlaces de rádio, tanto para ambientes internos quanto para ambientes externos.
	
	\item Capítulo 3: apresenta as principais características das redes Wi-Fi, a sua arquitetura, o padrão IEEE 802.11 (o qual as redes Wi-Fi são estruturadas) e, por fim, o método de inspeção \textit{site survey} como ferramenta para o planejamento e avaliação de redes \textit{wireless}.
	
	\item Capítulo 4: apresenta um estudo de caso realizado a partir da aplicação do \textit{site survey} na infraestrutura de rede sem fio do Bloco Didático do Instituto Federal de Educação, Ciência e Tecnologia do Ceará \textit{campus} Tauá; Expõe e discute os resultados obtidos ao fim do processo de coleta de dados do local alvo; Propõe medidas de intervenção na infraestrutura da rede para a melhoria da cobertura do sinal nos dois andares analisados.
	
	\item Capítulo 5: apresenta o encerramento do trabalho baseado no conteúdo discutido e nos resultados obtidos no estudo de caso.
\end{compactitem}